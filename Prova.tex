\documentclass{exam}
    \usepackage{ModeloIFBA}
    %\usepackage[alf]{abntex2cite}
    
    %Tecnologia em Análise e Desenvolvimento de Sistemas
    \renewcommand{\curso}{curso}
    %ILP001 - Introdução à Lógica de Programção
    \renewcommand{\materia}{materia}
    %ADS2017
    \renewcommand{\turma}{turma}
    %Leandro Costa Souza
    \renewcommand{\professor}{professor}
    %Prova 1
    \renewcommand{\avaliacao}{avaliacao}
    %02/10/2017
    \renewcommand{\dataavaliacao}{dataavaliacao}
    
    
    \newcommand{\duracao}{1h 40min}
    
    %\printanswers
    
    \begin{document}
    
        \FrameSep0pt	
\begin{framed}
	\begin{center}
		Orientações para a avaliação (\textbf{Duração}: \underline{\duracao})
	\end{center}
		\begin{multicols}{2}
		{\footnotesize
		\begin{enumerate}
		\itemsep0em
		\item Todos os códigos devem ser escritos utilizando a linguagem de programação java;
		\item Confira e assine sua avaliação;
		\item Na avaliação deverá ser utilizada caneta apenas nas cores azul ou preta;
		\item Questões respondidas à lápis não darão direito a recorreção;
		\item Respostas deverão ser escritas no local designado para cada questão;
		\item A interpretação das questões faz parte da avaliação;
		\item Em questões objetivas deve ser preenchido todo o espaço para a marcação;
		\item Não é permitido qualquer tipo de consulta;
		\item Não é permitida a troca de materiais entre colegas;
		\item Não é permitido o uso de equipamentos eletrônicos;
		\item Não é permitido destacar folhas da avaliação;
		\item Não é permitido conversa entre os avaliados.		
		\item Não é permitido utilizar folhas extras.
		\end{enumerate}
		}
		\end{multicols}
\end{framed}
    
        \begin{center}
            \gradetable[h][questions]
        \end{center}
    
        Serão avaliadas as aplicações dos conceitos de encapsulamento e uma correta modelagem considerando os conceitos de orientação à objeto.
        O código das questões são complementares.
        \begin{questions}
            \question[10] Construa as classes necessárias para um sistema de biblioteca que pode armazenar informações de itens emprestáveis. Esses itens podem ser \textit{livros} (título, autor e ISBN), \textit{apostilas} (título e autor) e midias \textit{audiovisuais} (título, autor e tipo de mídia(DVD, CD ...)).  A identificação dos livros deve ser por ISBN dos outros idem deve ser a junção de título e autor.
            \fillwithlines{\stretch{1}}

            \newpage
            \question[75]Considere o seguinte algoritmo: 
            \lstinputlisting[language=portugol]{code/calculo.alg}
            \begin{solutionorlines}[\stretch{1}]
                Resposta da questao q so ira aparecer quanto o parametro de printanswers for liberado
            \end{solutionorlines}	

            \question[25] Uma palavra de Fibonacci é definida pela função f(n) descrita abaixo. Esta sequência inicia com as seguintes palavras: b, a, ab, aba, abaab, abaababa, abaababaabaab, ... Faça uma função recursiva que receba um número N e retorne a N-ésima palavra de Fibonacci. 
            $$
             f(n)=\begin{cases}
                0, & \text{se $n=0$}\\
                1, & \text{se $n=1$}\\
                f(n-1) + f(n-2) , & \text{se $n>1$}
              \end{cases}
            $$
            Aqui $+$ denota a concatenação de duas strings
            
            %\fillwithlines{\stretch{1}}	
            \begin{solutionorlines}[\stretch{1}]
                resposta aqui	
            \end{solutionorlines}
            
            \newpage		
            \question[10] Selecione a opção que mostre quais são os conectivos lógicos que devem ser colocados nas expressões a seguir para que TODAS sejam verdadeiras. Considere A=2, B=3, C=6 e D=10.
            
            \begin{parts}
            \part (A+C < D-A) \_ (D-A = A+C)
            \part (B*A = C) \_ ((D div A) < C)
            \part ((D mod A) = (C mod B)) \_ ((B mod A) <> (C mod A))
            \part ((C div A) = B) \_ (B*C <= D+A*B)
            \part (((D div A)+B) > C+A) \_ ((D mod C) >= (A*A))
            \end{parts}
        
            \begin{choices}
                \choice ou – e – xou – ou – xou
                \choice ou – xou – ou – xou – e
                \choice e – e – ou – xou – ou
                \CorrectChoice xou – e – ou – xou – ou
                \choice ou – xou – e – xou – ou
            \end{choices}


            \newpage		
            \begin{center}
             \textbf{Rscunho}
            \end{center}	
            \fillwithlines{\stretch{1}}
            \newpage
            \fillwithlines{\stretch{1}}
    
        \end{questions}
    
        %\bibliography{biblio}
    
    \end{document}
    
    